\section{Introduction}

[Just some notes for now...]

Lots of backends for functional languages,
e.g. STG~\cite{evalpush,stg,llvm-haskell}, ABC~\cite{abc-machine}.
But they aren't simple enough that they are easy to bolt on to a new
language. Either too low level, or an interface isn't exposed, or
where an interface is exposed, there are constraints on the type
system. So things like Agda~\cite{norell-thesis} have resorted to
generating Haskell with unsafeCoerce, Cayenne~\cite{cayenne-icfp} used LML
with the type checker switched off. This works but we can't expect
GHC optimisations without working very hard, are limited to GHC's
choice of evaluation order, and could throw away useful information
gained from the type system.

Epic originally written for Epigram~\cite{levitation} (the
name\footnote{Coined by James McKinna} is
short for ``\textbf{Epi}gram \textbf{C}ompiler''). Now used by
Idris~\cite{idris-plpv}, also as an experimental back end for Agda.
It is specifically designed for reuse by other languages (in constrast
to, say, GHC Core).

\subsection{Features and non-features}

Epic will handle the following:

\begin{itemize}
\item Managing closures and thunks
\item Lambda lifting
\item Some optimisations (currently inlining, a supercompiler is planned)
\item Marshaling values to and from foreign functions
\item Garbage collection
\item Name choices (optionally)
\end{itemize}

\noindent
Epic will not do the following, by design:

\begin{itemize}
\item Type checking (no assumptions are made about the type system of
  the high level language being compiled)
\end{itemize}

Epic has few high level language features, but some additions will be
considered which will not compromise the simplicity of the core
language. For example, a pattern matching compiler is planned, and
primitives for parallel execution.

Also lacking, but entirely possible to add later (with some care) are
unboxed types.

\subsection{Why an Intermediate Language}

Why not generate Haskell, OCaml, Scheme, \ldots? In general they are
too high level and impose design choices and prevent certain
implementation choices. An intermediate level language such as Epic
allows the following:

\begin{description}
\item[Control of generated code]
A higher level target language imposes implementation choices such as
evaluation strategy and purity. Also makes it harder to use lower
level features where it might be appropriate (e.g. while loops, mutation).

\item[Control of language design]
Choice of a high level target language (especially a typed one) might
influence our type system design, restrict our choices for ease of
code generation. 

\item[Efficiency]
We might expect using a mature target language to give us
optimisations for free. This might be true in many cases, but only if
our source language is similar enough. e.g. in Epigram the type system
tells us more about the code than we can pass on to a Haskell back
end. 

\end{description}

Epic aims to provide the necessary features for implementing the
back-end of a functional language --- thunks, closures, algebraic data
types, scope management, lambda lifting --- without imposing
\remph{any} design choices on the high level language designer, with
the obvious exception that a functional style is encouraged!
A further advantage of Epic is that the library provides
\remph{compiler combinators}, which guarantee that any
output code will be syntactically correct and well-scoped.
