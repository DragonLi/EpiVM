%\documentclass{llncs}
\documentclass[orivec,dvips,10pt]{llncs}

\usepackage[draft]{comments}
%\usepackage[final]{comments}
% \newcommand{\comment}[2]{[#1: #2]}
\newcommand{\khcomment}[1]{\comment{KH}{#1}}
\newcommand{\ebcomment}[1]{\comment{EB}{#1}}

\usepackage{epsfig}
\usepackage{path}
\usepackage{url}
\usepackage{amsmath,amssymb} 
\usepackage{fancyvrb}

\newenvironment{template}{\sffamily}

\usepackage{graphics,epsfig}
\usepackage{stmaryrd}

\input{./macros.ltx}
\input{./library.ltx}

\NatPackage
\FinPackage

\newcounter{per}
\setcounter{per}{1}

\newcommand{\Ivor}{\textsc{Ivor}}
\newcommand{\Idris}{\textsc{Idris}}
\newcommand{\Funl}{\textsc{Funl}}
\newcommand{\Agda}{\textsc{Agda}}
\newcommand{\LamPi}{$\lambda_\Pi$}

\newcommand{\perule}[1]{\vspace*{0.1cm}\noindent
\begin{center}
\fbox{
\begin{minipage}{7.5cm}\textbf{Rule \theper:} #1\addtocounter{per}{1}
\end{minipage}}
\end{center}
\vspace*{0.1cm}
}

\newcommand{\mysubsubsection}[1]{
\noindent
\textbf{#1}
}
\newcommand{\hdecl}[1]{\texttt{#1}}

\begin{document}

\title{Epic --- A Library for Generating Compilers}
\author{Edwin Brady}

\institute{University of St Andrews, KY16 9SX, Scotland/UK,\\
\email{eb@cs.st-andrews.ac.uk}}

\maketitle

\begin{abstract}
Compilers for functional languages, whether strict or non-strict,
typed or untyped, need to handle many of the same problems, for
example thunks, lambda lifting, optimisation, garbage collection, and
system interaction.  Although implementation techniques are by now
well understood, it remains difficult for a new functional language to
exploit these techniques without either implementing a compiler from
scratch, or attempting fit the new language around another existing
compiler.  Epic is a compiled functional language which exposes
functional compilation techniques to a language implementor, with a
Haskell API. In this paper we describe Epic and outline how it may be
used to implement a high level language compiler, illustrating our
approach by implementing compilers for the $\lambda$-calculus and a
dynamically typed graphics language.

\end{abstract}

\section{Introduction}

[Just some notes for now...]

Lots of backends for functional languages,
e.g. STG~\cite{evalpush,stg,llvm-haskell}, ABC~\cite{abc-machine}.
But they aren't simple enough that they are easy to bolt on to a new
language. Either too low level, or an interface isn't exposed, or
where an interface is exposed, there are constraints on the type
system. So things like Agda~\cite{norell-thesis} have resorted to
generating Haskell with unsafeCoerce, Cayenne~\cite{cayenne-icfp} used LML
with the type checker switched off. This works but we can't expect
GHC optimisations without working very hard, are limited to GHC's
choice of evaluation order, and could throw away useful information
gained from the type system.

Epic originally written for Epigram~\cite{levitation} (the
name\footnote{Coined by James McKinna} is
short for ``\textbf{Epi}gram \textbf{C}ompiler''). Now used by
Idris~\cite{idris-plpv}, also as an experimental back end for Agda.
It is specifically designed for reuse by other languages (in constrast
to, say, GHC Core).

\subsection{Features and non-features}

Epic will handle the following:

\begin{itemize}
\item Managing closures and thunks
\item Lambda lifting
\item Some optimisations (currently inlining, a supercompiler is planned)
\item Marshaling values to and from foreign functions
\item Garbage collection
\item Name choices (optionally)
\end{itemize}

\noindent
Epic will not do the following, by design:

\begin{itemize}
\item Type checking (no assumptions are made about the type system of
  the high level language being compiled)
\end{itemize}

Epic has few high level language features, but some additions will be
considered which will not compromise the simplicity of the core
language. For example, a pattern matching compiler is planned, and
primitives for parallel execution.

Also lacking, but entirely possible to add later (with some care) are
unboxed types.

\subsection{Why an Intermediate Language}

Why not generate Haskell, OCaml, Scheme, \ldots? In general they are
too high level and impose design choices and prevent certain
implementation choices. An intermediate level language such as Epic
allows the following:

\begin{description}
\item[Control of generated code]
A higher level target language imposes implementation choices such as
evaluation strategy and purity. Also makes it harder to use lower
level features where it might be appropriate (e.g. while loops, mutation).

\item[Control of language design]
Choice of a high level target language (especially a typed one) might
influence our type system design, restrict our choices for ease of
code generation. 

\item[Efficiency]
We might expect using a mature target language to give us
optimisations for free. This might be true in many cases, but only if
our source language is similar enough. e.g. in Epigram the type system
tells us more about the code than we can pass on to a Haskell back
end. 

\end{description}

Epic aims to provide the necessary features for implementing the
back-end of a functional language --- thunks, closures, algebraic data
types, scope management, lambda lifting --- without imposing
\remph{any} design choices on the high level language designer, with
the obvious exception that a functional style is encouraged!
A further advantage of Epic is that the library provides
\remph{compiler combinators}, which guarantee that any
output code will be syntactically correct and well-scoped.


\section{The Epic Language}

Epic is based on the $\lambda$-calculus with some extensions.
It supports primitives such as strings and integers, as well as tagged
unions. There are additional control structures for specifying
evaluation order, primitive loop constructs, and calling foreign
functions. Foreign function calls are annotated with types, to assist
with marshaling values between Epic and C, but otherwise there are no
type annotations and there is no type checking --- as Epic is intended
as an intermediate language, it is assumed that the high level
language has already performed any necessary type checking. The
abstract syntax of the core language is given in Figure \ref{epicsyn}.
As a shorthand, we use de Bruijn telescope notation, $\tx$, to denote
a sequence of $\vx$.

\newcommand{\Con}[2]{\DC{Con}\:#1(#2)}

\FFIG{
\AR{
\begin{array}{rcll}\\
\vp & ::= & \vec{\VV{def}} & \mbox{(Epic program)} \\
\VV{def} & ::= & \vx(\tx) = \vt &
\mbox{(Top level definition)} \\
\\
\vt & ::= & \vx & \mbox{(Variable)} \\
& \mid &  \vt(\ttt) & \mbox{(Function application)} \\
& \mid & \lambda\vx\SC\vt & \mbox{(Lambda binding)} \\
& \mid & \RW{let}\:\vx\:=\:\vt\:\RW{in}\:\vt & \mbox{(Let
  binding)} \\
& \mid & \Con{\vi}{\ttt} & \mbox{(Constructor application)} \\
& \mid & \vt ! \vi & \mbox{(Argument projection)} \\
& \mid & \vt\:\VV{op}\:\vt & \mbox{(Infix operator)} \\
& \mid & \RW{if}\:\vt\:\RW{then}\:\vt\:\RW{else}\:\vt & \mbox{(Conditional)}\\
& \mid & \RW{case}\:\vt\:\RW{of}\:\vec{\VV{alt}} & \mbox{(Case expressions)}\\
& \mid & \RW{lazy}(\vt) & \mbox{(Lazy evaluation)} \\
& \mid & \RW{effect}(\vt) & \mbox{(Evaluate an effectful term)} \\
& \mid & \RW{while}\:\vt\:\vt & \mbox{(While loops)} \\
& \mid & \vx := \:\vt\:\RW{in}\:\vt & \mbox{(Variable update)} \\
& \mid & \RW{foreign}\:\vT\:\VV{str}\:\vec{(\vt\Hab\vT)} & \mbox{(Foreign call)} \\
& \mid & \RW{malloc}\:\vt\:\vt & \mbox{(Manual allocation)} \\
& \mid & \vi \mid \vf \mid \vc \mid \vb \mid \VV{str} & \mbox{(Constants)} \\
\\
\VV{alt} & ::= &
\Con{\vi}{\tx} \cq \vt & \mbox{(Constructors)}\\
& \mid & \vi \cq \vt & \mbox{(Integer constants)} \\
& \mid & \RW{default} \cq \vt & \mbox{(Match anything)} \\
\end{array}
\medskip
\\
\begin{array}{rcll}
\VV{op} & ::= & + \mid - \mid \times \mid / \mid\:
==\: \mid \:<\: \mid \:\le\: \mid \:>\: \mid \:\ge \: \mid \: << \:
\mid \: >>\\
\end{array}
\medskip
\\
\begin{array}{rcll}
\vx & ::= & \mbox{Variable name} \\
\vi & ::= & \mbox{Integer literal} \\
\vf & ::= & \mbox{Floating point literal} \\
\vc & ::= & \mbox{Character literal} \\
\vb & ::= & \mbox{Boolean literal} \: \DC{True} \mid \DC{False} \\
\VV{str} & ::= & \mbox{String literal} \\
\end{array}
\medskip
\\
\begin{array}{rcll}
\vT & ::= & \TC{Int} \mid \TC{Char} \mid \TC{Bool} \mid \TC{Float}
\mid \TC{String} & \mbox{(Primitives)} \\
 & \mid & \TC{Unit} & \mbox{(Unit type)} \\
 & \mid & \TC{Ptr} & \mbox{(Foreign pointers)} \\
% & \mid & \TC{Fun} & \mbox{(Any function type)} \\
% & \mid & \TC{Data} & \mbox{(Any data type)} \\
 & \mid & \TC{Any} & \mbox{(Polymorphic type)} \\
\end{array}
}
}
{Epic syntax}
{epicsyn}

\subsection{Definitions}

An Epic program consists of a sequence of \remph{untyped} function
definitions, with zero or more arguments. The entry point is the
function $\VV{main}$, which takes no arguments. For example:

\DM{
\begin{array}{ll}
\VV{factorial}(\vx)&\:=\:
\RW{if}\:\vx==0\:\AR{\RW{then}\:1\\
\RW{else}\:\vx\:\times\:\VV{factorial}(\vx-1)}
\medskip
\\
\VV{main}()&\:=\:\VV{putStrLn}(\VV{intToString}(\VV{factorial}(10)))
\end{array}
} 

\noindent
The right hand side of a definition is an expression consisting of
function applications, operators (arithmetic, comparison, and
bit-shifting), bindings and control structures (some low level and
imperative).  Functions may be \remph{partially applied}, i.e. applied
to fewer arguments than they require. Evaluating a partially applied
function results in a function which expects the remaining arguments.

\subsubsection*{Values}

Values in an Epic program are either one of the primitives (an
integer, floating point number, character, boolean or string) or a
\remph{tagged union}. Tagged unions are of the form $\DC{Con}
\vi(\vt_1,\ldots,\vt_n)$, where $\vi$ is the \remph{tag} and the $\ta$
are the \remph{fields}. The name $\DC{Con}$ is to suggest
``Constructor''. For example, we could represent a list using tagged
unions:

\begin{itemize}
\item $\DC{Con}\:0()$ representing the empty list.
\item $\DC{Con}\:1(\vx,\:\vxs)$ representing a cons cell, where $\vx$
  is the element and $\vxs$ is the tail of the list.
\end{itemize}

Tagged unions are inspected either using field projection ($\vt!\vi$
projects the $\vi$th field from a tagged union $\vt$) or by case
analysis. For example, to append two lists:

\DM{
\AR{
\VV{append}(\vxs,\vys)\:=
\RW{case}\:\vxs\:\RW{of}\\
\hg\hg\begin{array}{ll}
\DC{Con}\:0()&\cq\:\vys\\
\DC{Con}\:1(\vx,\vxs')&\cq\:\DC{Con}\:1(\vx,\VV{append}(\vxs',\vys))
\end{array}
}
}

\subsubsection*{Evaluation Strategy}

By default, expressions are evaluated eagerly (in applicative order),
i.e. arguments to functions and tagged unions are evaluated
immediately, left to right. Evaluation can instead be delayed using
the $\RW{lazy}$ construct. An expression $\RW{lazy}(\vt)$ will not be
evaluated until it is required by one of:

\begin{itemize}
\item Inspection in a $\RW{case}$ expression or the condition in an
  $\RW{if}$ statement.
\item Field projection.
\item Being passed to a foreign function.
\item Explicit evaluation with $\RW{effect}$. This evaluates
  side-effecting code (and does not update the thunk).
\end{itemize}

\noindent
Using $\RW{lazy}$ we can build an infinite list of values, ensuring
that a value will not be added to the list until it is needed:

\DM{
\AR{
\VV{countFrom}(\vx)\: = \:\DC{Cons}\:1(\vx,\:\RW{lazy}(\VV{countFrom}(\vx+1)))
}
}

We can safely write a function which takes the first $\vn$ values from
such a list, since the tail of the list will not be evaluated until it
is needed:

\DM{
\AR{
\VV{take}(\vn,\vxs)\: =\: \RW{case}\:\vxs\:\RW{of}\\
\hg\begin{array}{ll}
\DC{Con}\:0() & \cq\: \DC{Con}\:0() \\
\DC{Con}\:1(\vx,\vxs') & \cq\: \AR{
\RW{if}\:\vn == 0\:
\RW{then}\:\DC{Con}\:0() \\
\hg\RW{else}\:\DC{Con}\:1(\vx,\VV{take}(\vn-1, \vxs'))
}
\end{array}          
\medskip
\\
\VV{vals}()\:=\:\VV{take}(10, \VV{countFrom}(1))
}
}

\subsubsection*{Higher order functions}

Finally, expressions may contain $\lambda$ and $\RW{let}$
bindings. Higher order functions such as $\VV{map}$ are also permitted:

\DM{
\AR{
\VV{map}(\vf,\vxs)\:=\:
\RW{case}\:\vxs\:\RW{of}\\
\hg\hg\begin{array}{ll}
\DC{Con}\:0() & \cq\:\DC{Con}\:0()\\
\DC{Con}\:1(\vx,\vxs') & \cq\:\DC{Con}\:1(\vf(\vx), \VV{map}(\vf,\vxs'))
\end{array}
\medskip
\\
\VV{evens}(\vn)\:=\:\RW{let}\:\AR{\VV{nums}\:=\:
\VV{take}(\vn, \VV{countFrom}(1))\:\RW{in}\\
\VV{map}(\lambda\vx\SC\vx\:\times \:2, \VV{nums})
}
}
}

\subsection{Foreign Functions}

Most programs eventually need to interact with the operating
system. Epic provides a lightweight foreign function interface which
allows interaction with external C code. Each primitive type
corresponds to a C type, as shown in Table \ref{epicctypes}.  Most of
the conversions are straightforward mappings between the
languages. Otherwise, $\TC{Bool}$ is represented by an C \texttt{int},
and $\TC{Unit}$ is used for the return type of \texttt{void}
functions. $\TC{Ptr}$ is the type of pointer values originating in
C. Values can also be passed in their Epic representation
(\texttt{Closure*}) but the details are beyond the scope of this paper.

\begin{table}
\begin{center}
\begin{tabular}{|l|l|}
\hline
Epic type & C Type \\
\hline
$\TC{Int}$  & \texttt{int} \\
$\TC{Char}$ & \texttt{char} \\
$\TC{Bool}$ & \texttt{int} \\
$\TC{Float}$ & \texttt{double} \\
$\TC{String}$ & \texttt{char*} \\
$\TC{Ptr}$ & \texttt{void*} \\
$\TC{Unit}$ & \texttt{void} \\
$\TC{Any}$ & \texttt{Closure*} \\
\hline
\end{tabular}
\end{center}
\caption{Epic to C type conversion}
\label{epicctypes}
\end{table}

Epic values are represented differently from C values, so argument
values must be converted before making a foreign call, and the return
value converted back. Since Epic does no type checking or inference,
a foreign call requires the argument and return types to be given
explicitly. For example, consider the C sine function:

\begin{SaveVerbatim}{csin}

double sin(double x);

\end{SaveVerbatim}
\useverb{csin}

\noindent
We can call this function from Epic by giving the C name, the return
type (an Epic $\TC{Float}$) and the argument type (also an Epic
$\TC{Float}$).

\DM{
\VV{sin}(\vx)\:=\:\RW{foreign}\:\TC{Float}\:\mathtt{"sin"}\:(\vx\Hab\TC{Float})
}

\noindent
Often, dealing with foreign functions requires handling data types
which originate in C. For example, opening a file uses the
\texttt{fopen} function, which returns a pointer to the C type
\texttt{FILE}:

\begin{SaveVerbatim}{cfopen}

FILE* fopen(char* filename, char* mode); 

\end{SaveVerbatim}
\useverb{cfopen}

\noindent
We can use this function directly from Epic, using the $\TC{Ptr}$
type:

\DM{
\VV{fopen}(\vf,\vm)\:=\:\RW{foreign}\:\TC{Ptr}\:\mathtt{"fopen"}(\vf:\TC{String},
\vm:\TC{String})
}

\noindent
The $\TC{String}$ annotations in the foreign call ensure that the Epic
string representation will be passed to C as a \texttt{char*}. Other
file operations, such as writing or closing the file can refer
directly to the file handle returned as a $\TC{Ptr}$: 

\DM{
\begin{array}{ll}
\VV{fputs}(\vx,\vh) & =\:\RW{foreign}\:\TC{Unit}\:\mathtt{"fputs"}(\vx:\TC{String},\vh:\TC{Ptr})\\
\VV{fclose}(\vh) & =\:\RW{foreign}\:\TC{Unit}\:\mathtt{"fclose"}(\vh:\TC{Ptr})\\
\end{array}
}

\noindent
Many foreign functions have side effects (in particular, I/O). Epic
assumes that the high level language will arrange for side-effecting
functions to be evaluated in the correct order. For this reason, it is
important to understand the default evaluation strategy.

\subsubsection*{A note on memory management}

When working with foreign functions it is important to consider how
external code interacts with the garbage collector. In particular,
\remph{copying} garbage collectors can be problematic because there is
a danger that a value referenced by external code may be moved, and
there is no way for the collector to identify and move every reference
in this code. This could cause a problem if the external library uses
callbacks to Epic functions.

The current implementation uses a non-moving collector, the Boehm
conservative collector for C~\cite{boehm-gc}, thus avoiding any
problems. However, this is not a long term solution, as this kind of
collector does not generally perform well for programs which create a
lot of short lived objects (of which most functional programs are an
example). Any future garbage collector for Epic will need to ensure
that values originating in Epic are not copied once passed to foreign
code.

\subsection{Low Level Features}

Epic emphasises control over safety, and therefore provides some low
level features to give language implementations more control over
generated code.  A high level language may wish to use these features
in some performance critical contexts, whether for sequencing side
effects, implementing optimisations, or to provide run-time support
code.

\subsubsection*{Imperative features}

Epic allows sequencing, $\RW{while}$ loops and variable update. Since
the default evaluation strategy is strict, sequencing can be achieved
simply with \RW{let} bindings, discarding the variable:

\DM{
\begin{array}{llll}
\VV{main}\:= &
\RW{let}\:\_ & = &
\VV{putStr}(\mbox{\texttt{"Enter your name: "}})\:\RW{in}\\
& \RW{let}\:\vn & = &
\VV{getStr}()\:\RW{in} \\
& & & \VV{putStrLn}(\VV{append}(\mbox{\texttt{"Hello "}}, \vn))\\
\end{array}\\
}

\noindent
We can also use \RW{while} loops, and update variables. Variable
update ($\vx\: :=\:\vt_1\:\RW{in}\:\vt_2$) behaves like a $\RW{let}$
binding, except that $\vx$ must already be in scope, and the previous
value of $\vx$ is updated to be $\vt_2$. We can use these to write an
imperative program such as the following, which prints numbers from 1
to 10:

\DM{
\VV{main}\:=\:\RW{let}\:\AR{
\vc\:=\:0\:\RW{in}\\
\RW{while}\:(\vc<10)\:\\
\hg(\vc:=\vc+1\:\RW{in}\\
\hg\:\VV{putStrLn}(\VV{intToSring}(\vc)))
}
}

\subsubsection*{Memory allocation}

The $\RW{malloc}$ construct allows manual memory allocation. The
behaviour of $\RW{malloc}\:\vn\:\vt$ is to create a fixed pool of
$\vn$ bytes, and allocate only from this pool when evaluating
$\vt$. When evaluation is complete, the pool is freed, and the result
copied. This can be valuable where an upper bound on memory usage can
be predicted accurately (whether manually or automatically) as it
reduces the garbage collection overhead.

\subsection{Haskell API}

The primary interface to Epic is through a Haskell API, which is used
to build expressions and programs, and to compile them to executables.
Implementing a compiler for a high level language is then a matter of
converting the abstract syntax of a high level program into an Epic
program, through these ``compiler combinators'', and implementing any
run-time support as Epic functions.

\subsubsection*{Programs and expressions}

Build expressions, with a name supply. \texttt{Expr} is the internal
(abstract) representation of Epic expressions:

\begin{SaveVerbatim}{exprclass}

type Term = State Int Expr

class EpicExpr e where
    term :: e -> Term

\end{SaveVerbatim}
\useverb{exprclass}

Instances for raw expressions and terms. More interestingly, an
instance for functions which allows Haskell functions to be used to
build Epic functions without worring about scope. \texttt{R} and
\texttt{Lam} are internal representations for references and $\lambda$
bindings respectively.

\begin{SaveVerbatim}{exprinstance}

instance (EpicExpr e) => EpicExpr (Expr -> e) where
    term f = do var <- get
                put (var+1)
                let arg = MN "evar" var
                e' <- term (f (R arg))
                return (Lam arg e')

\end{SaveVerbatim}
\useverb{exprinstance}

\noindent
\textbf{Aside:}
Internally, names are separated into user supplied names, and names
invented by the machine. Using machine names guarantees we won't clash
with a user name, and we can give an annotation (\texttt{"evar"} here)
which says where the name arose.

\begin{SaveVerbatim}{epicnames}

data Name = UN String     -- user name
          | MN String Int -- machine generated name

\end{SaveVerbatim}
\useverb{epicnames}

It can be more convenient to give explicit names for $\lambda$
bindings, so we declare an instance to allow that:

\begin{SaveVerbatim}{exprnamed}

instance EpicExpr e => EpicExpr ([Name], e) where
    term (ns, e) = do 
        e' <- term e
        foldM (\e n -> return (Lam n e)) e' ns

\end{SaveVerbatim}
\useverb{exprnamed}

Both forms can be mixed freely.
A program is a collection of named Epic declarations:

\begin{SaveVerbatim}{eprogs}

data EpicDecl = forall e. EpicExpr e => EpicFn Name e
              | ...

type Program = [EpicDecl]

\end{SaveVerbatim}
\useverb{eprogs}

\noindent
Epic declarations are usually just a function (\texttt{EpicFn}) but
can also be used to include C header files, declare libraries for
linking and declare types for exporting as C types. The library also
provides a number of built in definitions for some common operations
such as outputting strings and converting data types:

\begin{SaveVerbatim}{bdefs}

basic_defs :: [EpicDecl]

\end{SaveVerbatim}
\useverb{bdefs}

We can compile a collection of definitions to an executable, or simply
execute them directly. Execution begins with the function called
\texttt{"main"} --- Epic reports an error if this function is not
defined:

\begin{SaveVerbatim}{compepic}

compile :: Program -> FilePath -> IO ()
run     :: Program -> IO ()

\end{SaveVerbatim}
\useverb{compepic}

\subsubsection*{Building expressions}

We've seen $\lambda$ bindings, using either Haskell's $\lambda$ or
pairing the names with their scope.

General form is that we build a \texttt{Term} (i.e. an expression
managing the name supply) by combining arbitrary Epic expressions
(i.e. instances of \texttt{EpicExpr}).

\begin{SaveVerbatim}{appapi}

(@@) :: (EpicExpr f, EpicExpr a) => f -> a -> Term

\end{SaveVerbatim}
\useverb{appapi}

Distinction between Haskell application and Epic
application (\texttt{@@}).

Operators. Separate versions for floating point and integer.

\begin{SaveVerbatim}{opsapi}

plus_, minus_, times_, divide_,    :: Op
plusF_, minusF_, timesF_, divideF_ :: Op
lt_, lte_, gt_, gte_,              :: Op
ltF_, lteF_, gtF_, gteF_,          :: Op
shiftl_, shiftr_                  :: Op

\end{SaveVerbatim}
\useverb{opsapi}

Convention: Epic keywords are represented by a Haskell function which
is the keyword with an underscore suffix. Arises because we can't have
``let'', ``if'', ``case'' etc as function names, and extended to other
primitives such as operators, foreign calls (anything that'd be a
keyword in general).

\begin{SaveVerbatim}{ifexp}

if_ :: (EpicExpr a, EpicExpr t, EpicExpr e) =>
       a -> t -> e -> Term

\end{SaveVerbatim}
\useverb{ifexp}

For $\RW{let}$ bindings, we can either use higher order syntax or bind
an explicit name:

\begin{SaveVerbatim}{letapi}

let_  :: (EpicExpr e) => 
         e -> (Expr -> Term) -> Term
letN_ :: (EpicExpr val, EpicExpr scope) =>
         Name -> val -> scope -> Term

\end{SaveVerbatim}
\useverb{letapi}

Shorthand if we're just using lets to sequence then throw away the
bound thing:

\begin{SaveVerbatim}{seqapi}

(+>) :: (EpicExpr c) => c -> Term -> Term
(+>) c k = let_ c (\x -> k)

\end{SaveVerbatim}
\useverb{seqapi}

Name management:

\begin{SaveVerbatim}{nameapi}

ref  :: Name -> Term
name :: String -> Name
fn   :: String -> Term

\end{SaveVerbatim}
\useverb{nameapi}

Building constructor forms. \texttt{tuple\_} is provided as a
shorthand if the tag is not important.

\begin{SaveVerbatim}{conapi}

con_   :: Int -> Term
tuple_ :: Term

\end{SaveVerbatim}
\useverb{conapi}

\subsubsection*{Case analysis}

Case expressions:

\begin{SaveVerbatim}{caseapi}

case_ :: (EpicExpr e) => e -> [Case] -> Term

\end{SaveVerbatim}
\useverb{caseapi}

Building case alternatives for constructors. We use the same trick as
we did for $\lambda$-bindings, either allowing Haskell to manage to
scope of constructor arguments, or giving names explicitly, or a mixture.

\begin{SaveVerbatim}{conalt}

class Alternative e where
    mkAlt :: Tag -> e -> Case

instance Alternative Expr
instance Alternative Term

instance Alternative e => Alternative (Expr -> e)
instance Alternative e => Alternative ([Name], e)

\end{SaveVerbatim}
\useverb{conalt}

We can build case alternatives for constructor forms, tuples, or
integer constants, as well as a default case if all other alternatives
fail to match. In each of the following, \texttt{e} is the right hand
side. For constructors and tuples, arguments may be bound in the
match. (We have to use \texttt{Alternative} rather than
\texttt{EpicExpr} to ensure that we get arguments bound in the match,
rather than a $\lambda$ binding on the right hand side).

\begin{SaveVerbatim}{altsapi}

con         :: Alternative e => Int -> e -> Case
tuple       :: Alternative e =>        e -> Case
constcase   :: EpicExpr e    => Int -> e -> Case
defaultcase :: EpicExpr e    =>        e -> Case

\end{SaveVerbatim}
\useverb{altsapi}

\subsubsection*{A complete example}

We have enough to write a simple example now. Here is the $\VV{map}$
function from earlier. Note the distinction between \texttt{con\_},
used to build a constructor, and \texttt{con}, used to introduce a
match alternative.

\begin{SaveVerbatim}{mapex}

map_ :: Expr -> Expr -> Term
map_ f xs = case_ xs 
   [con 0 (con_ 0),
    con 1 (\ (x :: Expr) (xs' :: Expr)
      -> con_ 1 @@ (f @@ x) @@ (fn "map" @@ f @@ xs'))]

\end{SaveVerbatim}
\useverb{mapex}

In the recursive call, we refer to the function by an Epic name,
rather than a Haskell name. So we need to ensure that these names are
consistent. We can declare that the name \texttt{"map"} refers to the
definition \texttt{map\_}:

\begin{SaveVerbatim}{mapdef}

mapDef :: EpicDecl
mapDef = EpicFn (name "map") map_

\end{SaveVerbatim}
\useverb{mapdef}

We have a function to add up all the integers in a list:

\begin{SaveVerbatim}{sumex}

sum_ :: Expr -> Term
sum_ xs = case_ xs
           [con 0 (int 0),
            con 1 (\ (x :: Expr) (xs' :: Expr) ->
                       op_ plus_ x (fn "sum" @@ xs'))]

\end{SaveVerbatim}
\useverb{sumex}

This constructs the Epic function:

\DM{
\AR{
\VV{sum}(\vxs)\:=\:
\RW{case}\:\vxs\:\RW{of}\\
\hg\hg\begin{array}{ll}
\DC{Con}\:0() & \cq\:0\\
\DC{Con}\:1(\vx,\vxs') & \cq\:\vx\:+\:\VV{sum}(\vxs')
\end{array}
}
}

We write a test funtion which doubles all the elements in a list, then
computes the sum of the result. Note that we have written the function
argument to map as an inline function. The main program simply outputs
the result of this function.

\begin{SaveVerbatim}{testex}

test_ = fn "sum" @@ 
       (fn "map" @@ \ (x :: Expr) -> op_ times_ x (int 2) 
                 @@ (con_ 1 @@ (int 5) @@
                    (con_ 1 @@ (int 10) @@ con_ 0)))

main_ = fn "putStrLn" @@ (fn "intToString" @@ fn "test")

\end{SaveVerbatim}
\useverb{testex}

\texttt{test\_} constructs the Epic function:

\DM{
\VV{test}\:=\:\VV{sum}(\VV{map}
\AR{
(\lambda\vx.x\:\times\:2, \\
\:\DC{Con}\:1(5,\DC{Con}\:1(10,\DC{Con}\:0()))))
}
}

Finally, we give a complete list of definitions, mapping concrete
names to the functions we've written.

\begin{SaveVerbatim}{alldefs}

defs = basic_defs ++ [EpicFn (name "main") main_, 
                      EpicFn (name "map") map_,
                      EpicFn (name "sum") sum_,
                      EpicFn (name "test") test_]

\end{SaveVerbatim}
\useverb{alldefs}

To compile and run this, we use the \texttt{run} function from the
Epic API, which builds an executable and runs it. The following
program outputs \texttt{30}.

\begin{SaveVerbatim}{mainex}

main = run defs

\end{SaveVerbatim}
\useverb{mainex}


\section{Example --- Compiling the $\lambda$-Calculus}

\label{sec:lc}

In this section we present a compiler for the untyped $\lambda$-calculus using
HOAS, showing the fundamental features of Epic
required to build a complete compiler.

%We have also implemented compilers
%for \LamPi{}~\cite{simply-easy}, a dependently typed language, which
%shows how Epic can handle languages with more expressive type systems,
%and a dynamically typed graphics language\footnote{\url{http://hackage.haskell.org/package/atuin}}, which shows how Epic can be
%used for languages with run-time type checking and which require
%foreign function calls.

\subsection{Representation}

Our example is an implementation of the untyped $\lambda$-calculus, plus
primitive integers and strings, and arithmetic and string operators. The
Haskell representation uses higher order abstract syntax (HOAS). We also
include global references (\texttt{Ref}) which refer to top level functions,
function application (\texttt{App}), constants (\texttt{Const}) and binary
operators (\texttt{Op}):

\begin{SaveVerbatim}{llang}

data Lang = Lam (Lang -> Lang)
          | Ref Name
          | App Lang Lang
          | Const Const
          | Op Infix Lang Lang

\end{SaveVerbatim}
\useverb{llang}

\noindent
Constants can be either integers or strings:

\begin{SaveVerbatim}{lconsts}

data Const = CInt Int | CStr String

\end{SaveVerbatim}
\useverb{lconsts}

\noindent
There are infix operators for arithmetic (\texttt{Plus},
\texttt{Minus}, \texttt{Times} and \texttt{Divide}), string
manipulation (\texttt{Append}) and comparison (\texttt{Eq},
\texttt{Lt} and \texttt{Gt}). The comparison operators return an
integer --- zero if the comparison is true, non-zero otherwise:

\begin{SaveVerbatim}{lops}

data Infix = Plus | Minus | Times | Divide | Append | Eq | Lt | Gt

\end{SaveVerbatim}
\useverb{lops}

\noindent
A complete program consists of a collection of named \texttt{Lang}
definitions:

\begin{SaveVerbatim}{lprogs}

type Defs = [(Name, Lang)]
\end{SaveVerbatim}
\useverb{lprogs}

\vspace*{0.5em}
\subsection{Compilation}

Our aim is to convert a collection of \texttt{Defs} into an
executable, using the \texttt{compile} or \texttt{run} function from
the Epic API.
Given an Epic \texttt{Program}, \texttt{compile} will generate an
executable, and \texttt{run} will generate an executable then run it.
Recall that a program is a collection of named Epic declarations:

\begin{SaveVerbatim}{eprogs}

data EpicDecl = forall e. EpicExpr e => EpicFn Name e
type Program = [EpicDecl]

\end{SaveVerbatim}
\useverb{eprogs}

\noindent
Our goal is to convert a \texttt{Lang} definition into
something which is an instance of \texttt{EpicExpr}. We use
\texttt{Term}, which is an Epic expression which carries a name
supply. Most of the term construction functions in the Epic API return
a \texttt{Term}.

\begin{SaveVerbatim}{buildtype}

build :: Lang -> Term

\end{SaveVerbatim}
\useverb{buildtype}

\noindent
The full implementation of \texttt{build} is given in Figure \ref{lcompile}.
In general, this is a straightforward traversal of the \texttt{Lang}
program, converting \texttt{Lang} constants to Epic constants,
\texttt{Lang} application to Epic application, and \texttt{Lang}
operators to the appropriate built-in Epic operators. 
                  
\begin{SaveVerbatim}{lcompile}
build :: Lang -> Term
build (Lam f)          = term (\x -> build (f (EpicRef x)))
build (EpicRef x)      = term x
build (Ref n)          = ref n
build (App f a)        = build f @@ build a
build (Const (CInt x)) = int x
build (Const (CStr x)) = str x
build (Op Append l r)  = fn "append" @@ build l @@ build r
build (Op op l r)      = op_ (eOp op) (build l) (build r)
    where eOp Plus   = plus_
          eOp Minus  = minus_
          ...
\end{SaveVerbatim}
\codefig{lcompile}{Compiling Untyped $\lambda$-calculus}

%\noindent
Using HOAS has the advantage that Haskell can
manage scoping, but the disadvantage that it is not straightforward to
convert the abstract syntax into another form. The Epic API also
allows scope management using HOAS, so we need to convert a function
where the bound name refers to a \texttt{Lang} value into a function
where the bound name refers to an Epic value. The easiest solution is
to extend the \texttt{Lang} datatype with an Epic reference:

\begin{SaveVerbatim}{lextend}

data Lang = ...
          | EpicRef Expr

build (Lam f) = term (\x -> build (f (EpicRef x)))

\end{SaveVerbatim}
\useverb{lextend}

\noindent
To convert a \texttt{Lang} function to an Epic function, we build an
Epic function in which we apply the \texttt{Lang} function to the Epic
reference for its argument. Every reference to a name in \texttt{Lang}
is converted to the equivalent reference to the name in Epic. 
Although it seems undesirable to extend \texttt{Lang} in this way, this
solution is simple to implement and preserves the
desirable feature that Haskell manages scope.
Compiling string append uses a built in function provided by the Epic
interface in \texttt{basic\_defs}:

\begin{SaveVerbatim}{lappend}

build (Op Append l r) = fn "append" @@ build l @@ build r

\end{SaveVerbatim}
\useverb{lappend}

\noindent
Given \texttt{build}, we can translate a collection of HOAS
definitions into an Epic program, add the built-in Epic definitions
and execute it directly. Recall that there must be a 
\textit{main} function or Epic will report an error --- we therefore add a
main function which prints the value of an integer expression
given at compile time.

\begin{SaveVerbatim}{lmain}

main_ exp = App (Ref (name "putStrLn"))
                (App (Ref (name "intToString")) exp)

mkProgram :: Defs -> Lang -> Program
mkProgram ds exp = basic_defs ++ 
                   map (\ (n, d) -> EpicFn n (build d)) ds ++
                   [(name "main", main_ exp)]

execute :: Defs -> Lang -> IO ()
execute p exp = run (mkProgram p exp)

\end{SaveVerbatim}
\useverb{lmain}

\noindent
Alternatively, we can generate an executable. Again, the entry point
is the Epic function \textit{main}:

\begin{SaveVerbatim}{lcomp}

comp :: Defs -> Lang -> IO ()
comp p exp = compile "a.out" (mkProgram p exp)

\end{SaveVerbatim}
\useverb{lcomp}

\noindent
This is a compiler for a very simple language, but a compiler for a
more complex language follows the same pattern: convert the abstract
syntax for each named definition into a named Epic \texttt{Term}, add
any required primitives (we have just used \texttt{basic\_defs} here),
and pass the collection of definitions to \texttt{run} or
\texttt{compile}. 




\section{Atuin --- A Dynamically Typed Graphics Language}

In this section we present a more detailed example language,
Atuin\footnote{\url{http://hackage.haskell.org/package/atuin}}, and
outline how to use Epic to implement a compiler for it. Atuin is a simple
imperative language with higher order procedures and dynamic type
checking, with primitive operations implementing turtle graphics.
The following example illustrates the basic features of the
language. The procedure \texttt{repeat} executes a code block a given
number of times:

\begin{SaveVerbatim}{turtleex1}

repeat(num, block) {
  if num > 0 {
     eval block
     repeat(num-1, block)
  }
}

\end{SaveVerbatim}
\begin{SaveVerbatim}{turtleex2}

polygon(sides, size, col) {
  if sides > 2 {
    colour col
    angle = 360/sides
    repeat(sides, {
      forward size
      right angle
    })
  }
}

\end{SaveVerbatim}

\useverb{turtleex1}

\noindent
Using \texttt{repeat}, \texttt{polygon} draws a polygon
with the given number of sides, a size and a colour:

\useverb{turtleex2}

\noindent
Programs consist of a number of procedure definitions, one of which
must be called \texttt{main} and take no arguments:

\begin{SaveVerbatim}{atuinmain}

main() {
  polygon(10,25,red)
}
\end{SaveVerbatim}
\useverb{atuinmain}

\subsection{Abstract Syntax}

The abstract syntax of Atuin is defined by algebraic data types
constructed by a Happy-generated parser. Constants can be one of four
types: integers, characters, booleans and colours:

\begin{SaveVerbatim}{consts}

data Const = MkInt Int   | MkChar Char
           | MkBool Bool | MkCol Colour

data Colour = Black | Red | Green | Blue | ...

\end{SaveVerbatim}
\useverb{consts}

\noindent
Atuin is an imperative language, consisting of sequences of commands
applied to expressions. We define expressions (\texttt{Exp})
and procedures (\texttt{Turtle}) mutually. Expressions can be constants
or variables, and combined by infix operators. Expressions can include
code blocks to pass to higher order procedures.

\begin{SaveVerbatim}{expr}

data Exp = Infix Op Exp Exp | Var Id
         | Const Const      | Block Turtle
\end{SaveVerbatim}
\useverb{expr}

\begin{SaveVerbatim}{opexpr}

data Op = Plus | Minus | Times  | Divide | ...

\end{SaveVerbatim}
\useverb{opexpr}

\noindent
Procedures define sequences of potentially side-effecting turtle
operations. There can be procedure calls, turtle commands, and some
simple control structures. \texttt{Pass} defines an empty code block:

\begin{SaveVerbatim}{turtle}

data Turtle = Call Id [Exp]     | Turtle Command
            | Seq Turtle Turtle | If Exp Turtle Turtle
            | Let Id Exp Turtle | Eval Exp
            | Pass

\end{SaveVerbatim}
\useverb{turtle}

\noindent
The turtle can be moved forward, turned left or right, or given a
different pen colour. The pen can also be raised, to allow the turtle
to move without drawing.

\begin{SaveVerbatim}{turtlecmd}

data Command = Fd Exp     | RightT Exp | LeftT Exp
             | Colour Exp | PenUp      | PenDown

\end{SaveVerbatim}
\useverb{turtlecmd}

\noindent
As with the $\lambda$-calculus compiler in Section \ref{sec:lc}, a
complete program consists of a collection of definitions,
where definitions include a list of formal parameters and the
program definition:

\begin{SaveVerbatim}{atuinprog}

type Proc = ([Id], Turtle)
type Defs = [(Id,  Proc)]
\end{SaveVerbatim}
\useverb{atuinprog}

\subsection{Compiling}

While Atuin is a different kind of language from the
$\lambda$-calculus, with complicating factors such as a global state
(the turtle), imperative features, and dynamic type checking, the
process of constructing a compiler follows the same general recipe, i.e.
define primitive operations as Epic functions, then convert each Atuin
definition into the corresponding Epic definition.

\subsubsection{Compiling Primitives}

The first step is to define primitive operations as Epic functions.
The language is dynamically typed, therefore we will need primitive
operations to check dynamically that they are operating on values of
the correct type. We define functions which construct Epic code for
building values, effectively using a single algebraic datatype to
capture all possible run-time values (i.e. values are
``uni-typed''~\cite{wadlerblame}).

\begin{SaveVerbatim}{valadt}

mkint  i = con_ 0 @@ i
mkchar c = con_ 1 @@ c
mkbool b = con_ 2 @@ b
mkcol  c = con_ 3 @@ c

\end{SaveVerbatim}
\useverb{valadt}

\noindent
Correspondingly, we can extract the concrete values safely from this
structure, checking that the value is the required type, e.g.

\begin{SaveVerbatim}{epicgetval}

getInt x  = case_ x [con 0 (\ (x :: Expr) -> x), 
                     defaultcase (error_ "Not an Int")]

\end{SaveVerbatim}
\useverb{epicgetval}

\noindent
Similarly, \texttt{getChar}, \texttt{getBool} and \texttt{getCol}
check and extract values of the appropriate type.
Using these, it is simple to define primitive arithmetic operations
which check that they are operating on the correct type, and report an
error if not.

\begin{SaveVerbatim}{primops}

primPlus   x y = mkint $ op_ plus_   (getInt x) (getInt y)
primMinus  x y = mkint $ op_ minus_  (getInt x) (getInt y)
primTimes  x y = mkint $ op_ times_  (getInt x) (getInt y)
primDivide x y = mkint $ op_ divide_ (getInt x) (getInt y)
\end{SaveVerbatim}
\useverb{primops}

\subsubsection{Graphics Operations}

We use the Simple DirectMedia Layer\footnote{\url{http://libsdl.org/}}
(SDL) to implement graphics operations. We implement C functions to
interact with SDL, and use Epic's foreign function interface to
call these functions. For example:

\begin{SaveVerbatim}{sdlglue}

void* startSDL(int x, int y);
void  drawLine(void* surf, int x, int y, int ex, int ey,
                           int r, int g, int b, int a);

\end{SaveVerbatim}
\useverb{sdlglue}

\noindent
The \texttt{startSDL} function opens a window with the given
dimensions, and returns a pointer to a \emph{surface} on which we can
draw; \texttt{drawLine} draws a line on a surface, between the given
locations, and in the given colour, specified as red, green, blue and
alpha channels.

We represent colours as a 4-tuple $(\vr,\vg,\vb,\va)$.  Drawing a line
in Epic involves extracting the red, green, blue and alpha components
from this tuple, then calling the C \texttt{drawLine} function. To
make a foreign function call, we use \texttt{foreign\_}, giving the C
function name and explicit types for each argument so that Epic
will know how to convert from internal values to C values:

\begin{SaveVerbatim}{drawline}

drawLine :: Expr -> Expr -> Expr -> Expr -> Expr -> Expr -> Term
drawLine surf x y ex ey col
    = case_ (rgba col)
        [tuple (\ r g b a ->
           foreign_ tyUnit "drawLine" 
             [(surf, tyPtr),
              (x, tyInt), (y, tyInt), (ex, tyInt), (ey, tyInt),
              (r, tyInt), (g, tyInt), (b, tyInt), (a, tyInt)]) ]

\end{SaveVerbatim}
\useverb{drawline}

\noindent
The turtle state is a tuple
$(\vs,\vx,\vy,\vd,\vc,\vp)$ where $\vs$ is a pointer to the SDL
surface, ($\vx$, $\vy$) gives the turtle's location, $\vd$ gives its
direction, $\vc$ gives the colour and $\vp$ gives the pen
state (a boolean, false for up and true for down). Note that this
state is not accessible by Atuin programs, so we do not dynamically check
each component.
To implement the \texttt{forward} operation, for example, we take the
current state, update the position according to the distance
given and the current direction, and if the pen is down, draws a line
from the old position to the new position.

\begin{SaveVerbatim}{drawfwd}

forward :: Expr -> Expr -> Term
forward st dist = case_ st 
  [tuple (\ (surf :: Expr) (x :: Expr) (y :: Expr) 
            (dir :: Expr) (col :: Expr) (pen :: Expr) -> 
    let_ (op_ plus_ x (op_ times_ (getInt dist) (esin dir)))
      (\x' -> let_ (op_ plus_ y (op_ timesF_ (getInt dist) (ecos dir)))
      (\y' -> if_ pen (fn "drawLine" @@ surf @@ x @@ y 
                                     @@ x' @@ y' @@ col) unit_ +>
              tuple_ @@ surf @@ x' @@ y' @@ dir @@ col @@ pen)))]

\end{SaveVerbatim}
\useverb{drawfwd}

\noindent
Here we have applied \texttt{getInt}, \texttt{esin} and
\texttt{ecos} as Haskell functions, so they will be inlined in the resulting Epic code.
In contrast, \texttt{drawLine} is applied as a separately defined Epic
function, using Epic's application operator (\texttt{@@}).

\vspace*{-1em}
\subsubsection{Compiling Programs}

Programs return an updated turtle state, and possibly perform 
side-effects such as drawing. An Atuin definition with
arguments $\va_1\ldots\va_n$ is translated to an Epic function
with a type of the following form:

\DM{
\vf \Hab \VV{State} \to \va_1 \to \ldots \to \va_n \to \VV{State}
}

\noindent
To compile a complete program, we add the primitive functions we have
defined above (line drawing, turtle movement, etc) to the list of
basic Epic definitions, and convert the user defined procedures to Epic.

\begin{SaveVerbatim}{prims}

prims = basic_defs ++ [EpicFn (name "initSDL") initSDL,
                       EpicFn (name "drawLine") drawLine,
                       EpicFn (name "forward") forward, ... ]

\end{SaveVerbatim}
\useverb{prims}

\noindent
We define a type class to capture conversion of expressions, commands
and full programs into Epic terms. Programs
maintain the turtle's state (an Epic \texttt{Expr}), and return a new
state, so we pass this state to the compiler.

\begin{SaveVerbatim}{compileclass}

class Compile a where
    compile :: Expr -> a -> Term

\end{SaveVerbatim}
\useverb{compileclass}

\noindent
In general, since we have set up all of the primitive operations as
Epic functions, compiling an Atuin program consists of directly
translating the abstract syntax to the Epic equivalent, making sure
the state is maintained. For example, to compile a call we
build an Epic function call and add the current state as the first
argument. Epic takes strings as identifiers, so we use \texttt{fullId
  :: Id -> String} to convert an Atuin identifier to an Epic identifier.

\begin{SaveVerbatim}{compfn}

compile state (Call i es) = app (fn (fullId i) @@ state) es
   where app f [] = f
         app f (e:es) = app (f @@ compile state e) es

\end{SaveVerbatim}
\useverb{compfn}

\noindent
Where operations are sequenced, we make sure that the state returned
by the first operation is passed to the next:

\begin{SaveVerbatim}{compseq}

compile state (Seq x y) 
   = let_ (compile state x) (\state' -> compile state' y)

\end{SaveVerbatim}
\useverb{compseq}

Atuin has higher order procedures which accept code blocks as
arguments. To compile a code block, we build a function which
takes the turtle state (that is, the state at the time the block is
executed, not the state at the time the block is created). 
Epic's
\texttt{effect\_} function ensures that a closure is evaluated, but
the result is not updated. Evaluating the closure may have side
effects which may need to be executed again --- consider the
\texttt{repeat} function above, for example, where the code block
should be evaluated on each iteration.

\begin{SaveVerbatim}{blockeval}

compile state (Block t) = term (\st -> compile st t)
compile state (Eval e)  = effect_ (compile state e @@ state)

\end{SaveVerbatim}
\useverb{blockeval}

\noindent
The rest of the operations are compiled by a direct mapping to the
primitives defined earlier. Finally, the main program sets up an SDL
surface, creates an initial turtle state, and passes that state to the
user-defined \texttt{main} function:

\begin{SaveVerbatim}{runmain}

init_turtle surf = tuple_ @@ surf @@ int 320 @@ int 240 @@ 
                                     int 180 @@ col_white @@ bool True

runMain :: Term
runMain = let_ (fn "initSDL" @@ int 640 @@ int 480)
          (\surface -> 
            (fn (fullId (mkId "main")) @@ (init_turtle surface)) +>
             flipBuffers surface +> pressAnyKey)

\end{SaveVerbatim}
\useverb{runmain}

\noindent
The full source code for Atuin and its compiler is available from
Hackage.


%\section{Implementation}

How it's implemented is not really important to the user --- a
compiler can target Epic without knowing, and we could drop in a new
back end at any time in principle.

There is currently one back end, but more are planned. Compiled via
C. Garbage collection with Boehm~\cite{boehm-gc},
\texttt{\%memory}. (Note that a non-moving collector makes things
easier for foreign functions, but may not be the best choice in the
long run).

Later plans: compile via LLVM, allow plug in garbage collectors
(important for embedded systems, device drivers, operating system
services, for example).



%\section{Performance}


\section{Related Work}


\section{Conclusion}



%\vspace{-0.2in}
%% \section*{Acknowledgments}

%% This work was partly funded by the Scottish Informatics and Computer
%% Science Alliance (SICSA) and by EU Framework 7 Project No. 248828
%% (ADVANCE). I thanks James McKinna, Kevin Hammond and Anil
%% Madhavapeddy for several helpful discussions, and the anonymous
%% reviewers for their constructive suggestions.

\bibliographystyle{abbrv}
\begin{small}
\bibliography{literature.bib}

\appendix

%\input{code}

\end{small}
\end{document}
