\section{The Epic Language}

Epic is based on the $\lambda$-calculus with some extensions.  It supports
primitives such as strings and integers, as well as tagged unions. There are
additional control structures for specifying evaluation order, primitive loop
constructs, and calling foreign functions. Foreign function calls are annotated
with types, to assist with marshaling values between Epic and C, but otherwise
there are no type annotations and there is no type checking --- as Epic is
intended as an intermediate language, it is assumed that the high level
language has already performed any necessary type checking. The abstract syntax
of the core language is given in Figure \ref{epicsyn}.  We use de Bruijn
telescope notation, $\tx$, to denote a sequence of $\vx$. Variable names are
represented by $\vx$, and $\vi$, $\vb$, $\vf$, $\vc$, and $\VV{str}$ represent
integer, boolean, floating point, character and string literals respectively.

\newcommand{\Con}[2]{\DC{Con}\:#1(#2)}

\FFIG{
\AR{
\begin{array}{rcllcll}\\
\vp & ::= & \vec{\VV{def}} & \mbox{(Epic program)} &
\VV{def} ::= & \vx(\tx) = \vt &
\mbox{(Definition)} \\
\\
\vt & ::= & \vx & \mbox{(Variable)}
& \mid &  \vt(\ttt) & \mbox{(Application)} \\
& \mid & \lambda\vx\SC\vt & \mbox{(Lambda binding)}
& \mid & \RW{let}\:\vx\:=\:\vt\:\RW{in}\:\vt & \mbox{(Let
  binding)} \\
& \mid & \Con{\vi}{\ttt} & \mbox{(Constructor)}
& \mid & \vt ! \vi & \mbox{(Projection)} \\
& \mid & \vt\:\VV{op}\:\vt & \mbox{(Infix operator)}
& \mid & \RW{if}\:\vt\:\RW{then}\:\vt\:\RW{else}\:\vt & \mbox{(Conditional)}\\
& \mid & \RW{case}\:\vt\:\RW{of}\:\vec{\VV{alt}} & \mbox{(Case expressions)}
& \mid & \RW{lazy}(\vt) & \mbox{(Lazy evaluation)} \\
& \mid & \RW{effect}(\vt) & \mbox{(Effectful term)}
& \mid & \RW{while}\:\vt\:\vt & \mbox{(While loops)} \\
& \mid & \vx := \:\vt\:\RW{in}\:\vt & \mbox{(Variable update)}
& \mid & \RW{foreign}\:\vT\:\VV{str}\:\vec{(\vt\Hab\vT)} & \mbox{(Foreign
call)} \\
& \mid & \RW{malloc}\:\vt\:\vt & \mbox{(Allocation)}
& \mid & \vi \mid \vf \mid \vc \mid \vb \mid \VV{str} & \mbox{(Constants)} \\
\\
\VV{alt} & ::= &
\Con{\vi}{\tx} \cq \vt & \mbox{(Constructors)}
& \mid & \vi \cq \vt & \mbox{(Constants)} \\
& \mid & \RW{default} \cq \vt & \mbox{(Match anything)} \\
\end{array}
\medskip
\\
\begin{array}{rcll}
\VV{op} & ::= & + \mid - \mid \times \mid / \mid\:
==\: \mid \:<\: \mid \:\le\: \mid \:>\: \mid \:\ge \: \mid \: << \:
\mid \: >>\\
\end{array}
\medskip
\\
\begin{array}{rcll}
\vT & ::= & \TC{Int} \mid \TC{Char} \mid \TC{Bool} \mid \TC{Float}
\mid \TC{String} & \mbox{(Primitives)} \\
 & \mid & \TC{Unit} & \mbox{(Unit type)} \\
 & \mid & \TC{Ptr} & \mbox{(Foreign pointers)} \\
% & \mid & \TC{Fun} & \mbox{(Any function type)} \\
% & \mid & \TC{Data} & \mbox{(Any data type)} \\
 & \mid & \TC{Any} & \mbox{(Polymorphic type)} \\
\end{array}
}
}
{Epic syntax}
{epicsyn}

\subsection{Definitions}

An Epic program consists of a sequence of \remph{untyped} function
definitions, with zero or more arguments. The entry point is the
function $\VV{main}$, which takes no arguments. For example:

\DM{
\begin{array}{ll}
\VV{factorial}(\vx)&\:=\:
\RW{if}\:\vx==0\:\AR{\RW{then}\:1\\
\RW{else}\:\vx\:\times\:\VV{factorial}(\vx-1)}
\medskip
\\
\VV{main}()&\:=\:\VV{putStrLn}(\VV{intToString}(\VV{factorial}(10)))
\end{array}
} 

\noindent
The right hand side of a definition is an expression consisting of
function applications, operators (arithmetic, comparison, and
bit-shifting), bindings and control structures (some low level and
imperative).  Functions may be partially applied.

\subsubsection*{Values}

Values in an Epic program are either one of the primitives (an
integer, floating point number, character, boolean or string) or a
\remph{tagged union}. Tagged unions are of the form $\DC{Con}\:
\vi(\vt_1,\ldots,\vt_n)$, where $\vi$ is the \remph{tag} and the $\ttt$
are the \remph{fields}; the name $\DC{Con}$ suggests
``Constructor''. For example, we could represent a list using tagged
unions, with $\DC{Con}\;0\:()$ representing the empty list and
$\DC{Con}\;1\:(\vx,\:\vxs)$ representing a cons cell, where $\vx$
is the element and $\vxs$ is the tail of the list.

Tagged unions are inspected either using field projection ($\vt!\vi$
projects the $\vi$th field from a tagged union $\vt$) or by case
analysis. E.g., to append two lists:

\DM{
\AR{
\VV{append}(\vxs,\vys)\:=
\RW{case}\:\vxs\:\RW{of}\\
\hg\hg\begin{array}{ll}
\DC{Con}\:0()&\cq\:\vys\\
\DC{Con}\:1(\vx,\vxs')&\cq\:\DC{Con}\:1(\vx,\VV{append}(\vxs',\vys))
\end{array}
}
}

\subsubsection*{Evaluation strategy}

By default, expressions are evaluated eagerly (in applicative order),
i.e. arguments to functions and tagged unions are evaluated
immediately, left to right. Evaluation can instead be delayed using
the $\RW{lazy}$ construct. An expression $\RW{lazy}(\vt)$ 
builds a thunk for the expression $\vt$ which will not be
evaluated until it needs to be inspected, typically by one of:
%\begin{itemize}
%\item 
inspection in a $\RW{case}$ expression or the condition in an
  $\RW{if}$ statement;
%\item 
field projection;
%\item 
being passed to a foreign function;
% FIXME: you haven't explained thunks yet...
%\item 
explicit evaluation with $\RW{effect}$.%\end{itemize}
An expression $\RW{effect}(\vt)$ evaluates the thunk $\vt$ \remph{without}
updating the thunk with the result. This is to facilitate evaluation of
side-effecting expressions.

\subsubsection*{Higher order functions}

Finally, expressions may contain $\lambda$ and $\RW{let}$
bindings. Higher order functions such as $\VV{map}$ are also permitted:

\DM{
\AR{
\VV{map}(\vf,\vxs)\:=\:
\RW{case}\:\vxs\:\RW{of}\\
\hg\hg\begin{array}{ll}
\DC{Con}\:0() & \cq\:\DC{Con}\:0()\\
\DC{Con}\:1(\vx,\vxs') & \cq\:\DC{Con}\:1(\vf(\vx), \VV{map}(\vf,\vxs'))
\end{array}
\medskip
\\
\VV{evens}(\vn)\:=\:\RW{let}\:\AR{\VV{nums}\:=\:
\VV{take}(\vn, \VV{countFrom}(1))\:\RW{in}\\
\VV{map}(\lambda\vx\SC\vx\:\times \:2, \VV{nums})
}
}
}

\subsection{Foreign Functions}

Most programs eventually need to interact with the operating
system. Epic provides a lightweight foreign function interface which
allows interaction with external C code.
Since Epic does no type checking or inference,
a foreign call requires the argument and return types to be given
explicitly. e.g. the C function:

\begin{SaveVerbatim}{csin}

double sin(double x);

\end{SaveVerbatim}
\useverb{csin}

\noindent
We can call this function from Epic by giving the C name, the return
type (an Epic $\TC{Float}$, which corresponds to a C \texttt{double})
and the argument type (also an Epic $\TC{Float}$).

\DM{
\VV{sin}(\vx)\:=\:\RW{foreign}\:\TC{Float}\:\mathtt{"sin"}\:(\vx\Hab\TC{Float})
}

\subsection{Low Level Features}

Epic emphasises control over safety, and therefore provides some low
level features. A high level language may wish to use these features
in some performance critical contexts, whether for sequencing side
effects, implementing optimisations, or to provide run-time support
code. 
Epic allows sequencing, $\RW{while}$ loops and variable update, and
provides a $\RW{malloc}$ construct for manual memory allocation (by default
memory is garbage collected). The
behaviour of $\RW{malloc}\:\vn\:\vt$ is to create a fixed pool of
$\vn$ bytes, and allocate only from this pool when evaluating
$\vt$. Due to space restrictions we will not discuss these constructs further. 

\subsection{Haskell API}

The primary interface to Epic is through a Haskell API which is used to build
expressions and programs with higher order abstract syntax (HOAS)~\cite{hoas}.
Implementing a compiler for a high level language then involves 
converting the abstract syntax of a high level program into an Epic program,
through these ``compiler combinators'', and implementing any run-time support
as Epic functions.

\subsubsection*{Programs and expressions}

The API allows the building of Epic programs with an Embedded Domain
Specific Language (EDSL) style interface, i.e. we try to exploit
Haskell's syntax as far as possible. There are several
possible representations of Epic expressions.
\texttt{Expr} is the internal abstract representation, and
\texttt{Term} is a representation which carries a name supply. We have
a type class \texttt{EpicExpr} which provides a function \texttt{term}
for building a concrete expression using a name supply:

\begin{SaveVerbatim}{exprclass}

type Term = State NextName Expr
class EpicExpr e where
    term :: e -> Term

\end{SaveVerbatim}
\useverb{exprclass}

\noindent
There are straightforward instances of \texttt{EpicExpr} for the
internal representations \texttt{Expr} and \texttt{Term}. There is also
an instance for \texttt{String}, which parses concrete syntax, which
is beyond the scope of this paper. More interestingly, we can build an
instance of the type class which allows Haskell functions to be used
to build Epic functions.  This means we can use Haskell names for Epic
references, and not need to worry about scoping or ambiguous name
choices. 

\begin{SaveVerbatim}{exprinstance}

instance EpicExpr e => EpicExpr (Expr -> e) where

\end{SaveVerbatim}
\useverb{exprinstance}

\noindent
Alternatively, function arguments can be given explicit names, constructed
with \texttt{name}. 
A reference to a name is built with \texttt{ref}, and
\texttt{fn} composes \texttt{ref} and \texttt{name}.

\begin{SaveVerbatim}{nameapi}

name :: String -> Name
ref  :: Name   -> Term
fn   :: String -> Term

\end{SaveVerbatim}

\useverb{nameapi}

\noindent
We provide an instance of \texttt{EpicExpr} to allow a user to give
names explicitly. This may be desirable when the abstract syntax for the 
high level language we are compiling has explicit names.

\begin{SaveVerbatim}{exprnamed}

instance EpicExpr e => EpicExpr ([Name], e) where

\end{SaveVerbatim}
\useverb{exprnamed}

\noindent
For example, the identity function can be built in either of the following ways:

\begin{SaveVerbatim}{ids}

id1, id2 :: Term
id1 = term (\x -> x)
id2 = term ([name "x"], fn "x")

\end{SaveVerbatim}
\useverb{ids}

\noindent
Both forms, using Haskell functions or explicit names, can be mixed
freely in an expression. A program is a collection of named Epic
expressions built using the \texttt{EpicExpr} class:

\begin{SaveVerbatim}{eprogs}

type Program = [EpicDecl]
data EpicDecl = forall e. EpicExpr e => EpicFn Name e

\end{SaveVerbatim}
\useverb{eprogs}

\noindent
The library provides a number of built-in definitions for some common
operations such as outputting strings and converting data types:

\begin{SaveVerbatim}{bdefs}

basic_defs :: [EpicDecl]

\end{SaveVerbatim}
\useverb{bdefs}

\noindent
In this paper we use $\VV{putStr}$ and $\VV{putStrLn}$ for outputting
strings, $\VV{append}$ for concatenating strings, and
$\VV{intToString}$ for integer to string conversion.
We can compile a collection of definitions to an executable, or simply
execute them directly. Execution begins with the function called
\textit{main} --- Epic reports an error if this function is not
defined:

\begin{SaveVerbatim}{compepic}

compile :: Program -> FilePath -> IO ()
run     :: Program -> IO ()
\end{SaveVerbatim}
\useverb{compepic}

\vspace*{-0.5em}
\subsubsection*{Building expressions}

We have seen how to build $\lambda$ bindings with the
\texttt{EpicExpr} class, using either Haskell's $\lambda$ or pairing
explicitly bound names with their scope. We now add further
sub-expressions.  The general form of functions which build
expressions is to create a \texttt{Term}, i.e. an expression which
manages its own name supply by combining arbitrary Epic
expressions, i.e. instances of \texttt{EpicExpr}. For example, to
apply a function to an argument, we provide an \texttt{EpicExpr} for
the function and the argument:

\begin{SaveVerbatim}{appapi}

infixl 5 @@
(@@) :: (EpicExpr f, EpicExpr a) => f -> a -> Term

\end{SaveVerbatim}
\useverb{appapi}

\noindent
Since \texttt{Term} itself is an instance of \texttt{EpicExpr}, we can
apply a function to several arguments through nested applications of
\texttt{@@}, which associates to the left as with normal Haskell
function application. We have several arithmetic operators, including
arithmetic, comparison and bitwise operators, e.g.:

\begin{SaveVerbatim}{opsapi}

op_ :: (EpicExpr a, EpicExpr b) => Op -> a -> b -> Term
plus_, minus_, times_, divide_ :: Op

\end{SaveVerbatim}
\useverb{opsapi}

\noindent
We follow the convention that Epic keywords and primitive
operators are represented by a Haskell function with an underscore
suffix. This convention arises because we cannot use Haskell keywords
such as \texttt{if}, \texttt{let} and \texttt{case} as function names.
For consistency, we have extended the convention to all functions and
operators. \texttt{if...then...else} expressions are built using the
\texttt{if\_} function:

\begin{SaveVerbatim}{ifexp}

if_ :: (EpicExpr a, EpicExpr t, EpicExpr e) => a -> t -> e -> Term

\end{SaveVerbatim}
\useverb{ifexp}

\noindent
For $\RW{let}$ bindings, we can either use HOAS or bind
an explicit name. To achieve this we implement a type
class and instances which support both:

\begin{SaveVerbatim}{letapi}

class LetExpr e where
    let_ :: EpicExpr val => val -> e -> Term

instance EpicExpr sc => LetExpr (Name, sc)
instance                LetExpr (Expr -> Term)

\end{SaveVerbatim}
\useverb{letapi}

%% \noindent
%% Occasionally, especially in imperative code, we use $\RW{let}$
%% bindings for sequencing actions. We provide a shorthand for this
%% common case:

%% \begin{SaveVerbatim}{seqapi}

%% (+>) :: (EpicExpr c) => c -> Term -> Term
%% (+>) c k = let_ c (\x -> k)

%% \end{SaveVerbatim}
%% \useverb{seqapi}

\noindent
To build a constructor form, we apply a constructor with an integer
\remph{tag} to its arguments. We build a constructor using the
\texttt{con\_} function, and provide a shorthand \texttt{tuple\_} for
the common case where the tag can be ignored --- as the name suggests,
this happens when building tuples and records:

\begin{SaveVerbatim}{conapi}

con_   :: Int -> Term
tuple_ :: Term
\end{SaveVerbatim}
\useverb{conapi}

\subsubsection*{Case analysis}

\noindent
To inspect constructor forms, or to deconstruct tuples, we use
$\RW{case}$ expressions. A case expression chooses one of the
alternative executions path depending on the value of the scrutinee,
which can be any Epic expression:

\begin{SaveVerbatim}{caseapi}

case_ :: EpicExpr e => e -> [Case] -> Term

\end{SaveVerbatim}
\useverb{caseapi}

\noindent
We leave the definition of \texttt{Case} abstract (although it is
simply an Epic expression carrying a name supply) and provide an
interface for building case branches.
The scrutinee is matched against each branch, in order. To match
against a constructor form, we use the same trick as we did for
$\lambda$-bindings, either allowing Haskell to manage the scope of
constructor arguments, or giving names explicitly to arguments, or a
mixture. For convenience, we make \texttt{Expr} and \texttt{Term} 
instances to allow empty argument lists.

\begin{SaveVerbatim}{conalt}

class Alternative e where
    mkAlt :: Tag -> e -> Case
\end{SaveVerbatim}
\begin{SaveVerbatim}{conalt2}

instance Alternative Expr
instance Alternative Term
\end{SaveVerbatim}
\begin{SaveVerbatim}{conalt3}

instance Alternative e => Alternative (Expr -> e)
instance Alternative e => Alternative ([Name], e)

\end{SaveVerbatim}
\useverb{conalt}

\useverb{conalt2}

\useverb{conalt3}

\noindent
We can build case alternatives for constructor forms (matching a
specific tag), tuples, or integer constants (matching a specific
constant), and a default case if all other alternatives fail to
match. In the following, \texttt{e} is an expression which
gives the argument bindings, if any, and the right hand side of the
match.

%(We have to use \texttt{Alternative} rather than
%\texttt{EpicExpr} to ensure that we get arguments bound in the match,
%rather than a $\lambda$ binding on the right hand side).

\begin{SaveVerbatim}{altsapi}

con         :: Alternative e => Int -> e -> Case
tuple       :: Alternative e =>        e -> Case
constcase   :: EpicExpr e    => Int -> e -> Case
defaultcase :: EpicExpr e    =>        e -> Case
\end{SaveVerbatim}
\useverb{altsapi}

